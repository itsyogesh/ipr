\documentclass[a4paper,11pt]{article}
\usepackage{graphicx}

\DeclareGraphicsExtensions{.pdf,.png,.jpg,.gif}

\newcommand{\HRule}{\rule{\linewidth}{0.5mm}}

%define the title page

\begin{document}

\begin{titlepage}
\begin{center}

\includegraphics[width=0.15\textwidth]{./nsitlogo}~\\[1cm]

\textsc{\LARGE Netaji Subhas Institute of Technology}\\[1.5cm]
\vfill
% Title
\HRule \\[0.4cm]
{\huge \bfseries Intellectual Property Rights \\[0.4cm] }

\HRule \\[1.5cm]

\vfill

% Submission details
\begin{minipage}[t]{0.4\textwidth}
\begin{flushleft} \large
\emph{Submitted By:} \\
Shashank Singh
\end{flushleft}
\end{minipage}
\begin{minipage}[t]{0.4\textwidth}
\begin{flushright} \large
\emph{Submitted to:} \\
Mr.~A.~V.~ Mulley
\end{flushright}
\end{minipage}

\vfill

% Bottom of the page
{\large \today}

\end{center}
\end{titlepage}

\newpage
\tableofcontents

\newpage

\section{Basic Definition}
Intellectual property (IP) refer to creations of the mind, such as inventions,
literary and artistic works, designs and symbols, names and images used in
commerce.

IP is protected in law by, for example \emph{patents}\cite{patents},
\emph{copyright}\cite{copyright} and \emph{trademarks}\cite{trademarks} which
enable people to earn recognition or financial benifit from what they invent or
create. By striking the right balance between the interests of innovators and
wider public interest, the IP system aims to foster an environment in which
creativity and innovation can flourish.

\section{History}
Although many of the legal principles governing IP and IPR have evolved over
centuries, it was not until the 19\textsuperscript{th} century that the term \emph{intellectual
property} began to be used, and not until the late 20\textsuperscript{th}
century that it became commonplace in the majority of the world.\cite{ipweb}
\emph{The Statue of Monopolies (1624)}\cite{statueofmonopolies} and the British
\emph{Statue of Anne (1710)}\cite{statueofanne} are now seen as the origin of
\emph{patent law}\cite{patentlaw} and \emph{copyright}\cite{copyrightlaw}
respectively,\cite{ipweb} firmly establishing the concept of intellectual
property.

The history of patents does not begin with inventions, but rather with royal
grants by \emph{Queen Elizabeth I (1558 - 1603}\cite{queenelizabethi} for monopoly privileges.
Approximately 200 years after the end of Elizabeth's reign, however a
patent represents a legal \emph{right}\cite{legalright} obtained by an inventor
providing for exclusive control over the production and sale of his mechanical
or scientific invention demonstrating the evolution of patents from royal
prerogative to common - law doctrine.

\section{Branches of Intellectual Property}
Intellectual Property is usually divided into two branches, namely
\emph{industrial property} and
\emph{copyright}\cite{copyrightlaw}.

\subsection{Industrial Property}
Industrial property typically consists of patents to protect inventions and
industrial designs, which are aesthetic creations determining the appearence of
industrial products. It also covers \emph{trademarks}\cite{trademarks}, trade names and desgnations, as
well as geographical indications, and protection against unfair competition.
They are described as follows:

\subsubsection{Trademarks}
A trademark is a sign capable of distinguishing the goods or services of one
enterprise from those of other enterprise. Trademarks are protected by
\emph{intellectual property}\cite{ipweb} rights.

\subsubsection{Trade names}
Trade names are used by profit and non - profit  entities, political and
religious organizations, industry and agriculture, manufacturers and producers,
wholesalers and retailers, partnerships and coorporations and a host of other
business association. A trade name may be the actual name of a given business
or an assumed name under which a business operates and holds itself out to the
public.\cite{tradename}

\subsubsection{Geographical indications}
A geographical indication (GI) is a name used on certain products which
corresponds to a specific geographical location or origin (e.g. a town, region,
or country). The use of GI may act as a certification that the product
possesses certain qualities, is made up according to traditional methods, or
enjoys a certain reputation, due to its geographical
origin.\cite{geographicalindications}

\subsection{Copyright}
Copyright relates to artistic creations, such as poems, novels, music,
paintings, and cinematographic works. In most European languages other than
English, copyright is known as author's right. The expression copyright refers to the main
act which, in respect of literary and artistic creations, may be made only by the
author or with his authorization. That act is the making of copies of the literary
or artistic work, such as a book, a painting, a sculpture, a photograph, or a
motion picture. The second expression, author’s rights refers to the person who 
is the creator of the artistic work, its author, thus underlining the fact, recognized
in most laws, that the author has certain specific rights in his creation, such as 
the right to prevent a distorted reproduction, which only he can exercise, whereas
other rights, such as the right to make copies, can be exercised by other persons,
for example, a publisher who has obtained a license to this effect from the author.

\section{How Intellectual Property Works}

\begin{figure}
\centering
\includegraphics[width=0.7\textwidth]{patent_office}
\caption{How Patent works\cite{patentworkjapan}}
\end{figure}

\subsection{Patents}
Patents, also referred to as patents for invention, are the most widespread
means
of protecting the rights of inventors. Simply put, a patent is the right
granted to
an inventor by a State, which
allows the inventor to exclude anyone else from commercially exploiting his
invention for a limited period, generally 20 years. By granting an exclusive
right,
patents provide incentives to individuals, offering them recognition for their
creativity and material reward for their marketable inventions. 

The word patent, or letters patent, also denotes the document issued by the
relevant
government authority. In order to obtain a patent for an invention, the
inventor,
or the entity he works for, submits an application to the national or regional
patent
office. In the application the inventor must describe the invention in detail
and
compare it with previous existing technologies in the same field in order to
demonstrate its newness. 

\subsubsection{Conditions of patentability}
Not all inventions are patentable. Laws generally require that an invention
fulfill the
following conditions, known as the requirements or \emph{conditions of
patentability}.

\flushleft
\begin{enumerate}
\begin{description}
\item[Industrial Applicability (utility)] The invention must be of practical
use, or capable of some kind of industrial application.
\item[Novelty] It must show some new characteristic that is not known in the
body of existing knowledge in its technical field.
\item[Intensive step(non-obviousness)] It must show an inventive step that
could not be deduced by a person with average knowledge of the technical field.
\item[Patentable subject matter] The invention must fall within the scope of
patentable subject matter as defined by national law. This varies from one
country to another.
\end{description}
\end{enumerate}

\subsection{Utility Models}
The expression “utility model” is simply a name given to a title of protection
for certain inventions, such as inventions in the mechanical field. Utility models are
usually sought for technically less complex inventions or for inventions that
have a short commercial life. The procedure for obtaining protection for a utility model is usually shorter and simpler than for obtaining a patent. Substantive and
procedural requirements under the applicable laws differ to a large extent
among the countries and regions that have a utility model system, but utility models
usually differ from patents for invention in the following main respects: 

\flushleft
\begin{enumerate}
\item The \emph{requirements} for accquiring a utility model are less stringent
than for patents. While the "novelty" requirement must always be met, that of
"inventive step" or "non-obviousness" may be much less or even absent
altogether. In practice, protection for utility models is often sought for
innovations of rather incremental nature, which may not meet the patentability
criteria.
\item The maximum \underline{term of protection} provided by law for a utility
model is generally shorter than the maximum term of protection provided for a
patent for invention (usually between 7 and 10 years)
\item The \emph{fees} required for obtaining and maintaining the right are
generally lower than those for patents. 
\end{enumerate}

\section{Objective of IP Laws}

\subsection{Financial incentive}
These exclusive rights allow owners of intellectual property to benifit from
the property they have created, providing a financial incentive for the
creation of an investment in intellectual property.

\subsection{Economic Growth}
The WIPO treaty and several related international agreements underline that the
protection of intellectual property rights is essential to maintaining economic
growth.

\clearpage
\newpage 

\begin{thebibliography}{14}

\bibitem{iprbook} Intellectual Property, Indigenous People and Their Knowledge
\textit{https://www.goodreads.com/book/show/19524983-intellectual-property-indigenous-people-and-their-knowledge}

\bibitem{patents} World Intellectual Property Organization
\textit{http://www.wipo.int/patents/en/}

\bibitem{tradename} The Free Dictionary
\textit{http://legal-dictionary.thefreedictionary.com/trade+name}

\bibitem{geographicalindications} The Free Dictionary
\textit{http://encyclopedia.thefreedictionary.com/Geographical+Indications}

\bibitem{patentworkjapan} Japanese Patent Office
\textit{http://www.slideshare.net/kammababu/patents-burrone} 

\bibitem{copyright} World Intellectual Property Organization
\textit{http://www.wipo.int/copyright/en/}

\bibitem{trademarks} World Intellectual Property Organization
\textit{http://www.wipo.int/trademarks/en/}

\bibitem{ipweb} Wikipedia - The free encyclopedia
\textit{http://en.wikipedia.org/wiki/Intellectual\textunderscore property}

\bibitem{statueofmonopolies} Wikipedia - The free encyclopedia
\textit{http://en.wikipedia.org/wiki/Statute\textunderscore of\textunderscore Monopolies}

\bibitem{statueofanne} Wikipedia - The free encyclopedia
\textit{http://en.wikipedia.org/wiki/Statute\textunderscore of\textunderscore Anne}

\bibitem{patentlaw} Wikipedia - The free encyclopedia
\textit{http://en.wikipedia.org/wiki/Patent}

\bibitem{copyrightlaw} Wikipedia - The free encyclopedia
\textit{http://en.wikipedia.org/wiki/Copyright}

\bibitem{queenelizabethi} Wikipedia - The free encyclopedia
\textit{http://en.wikipedia.org/wiki/Elizabeth\textunderscore I\textunderscore of\textunderscore England}

\bibitem{legalright} Wikipedia - The free encyclopedia \\
\textit{http://en.wikipedia.org/wiki/Rights}

\end{thebibliography}
\end{document}
